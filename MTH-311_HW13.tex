%This is a Latex file.
\documentclass[12pt]{article}
\usepackage{latexsym,fancyhdr,amsmath,amsfonts,amsthm,dsfont}
\usepackage{amssymb}

% margins are relative to the default of 1 in
%\topmargin       -0.2 in

\topmargin        -0.2 in
\textheight       8.4 in
\oddsidemargin    0 in     % this is for pages 1, 3, 5, ...
\evensidemargin   0 in     % and this for 2, 4, 6, ...
\textwidth        6.5 in
\headheight       0 in     % we won't have a running head, nor
\headsep          .35 in     % any extra space between head and text

%\parindent 0pt

\pagestyle{fancy} \lhead{\sf MTH 311} \chead{\sf Homework \#13}
\rhead{\sf Austin Klum} \lfoot{} \cfoot{} \rfoot{}

\newcommand{\C}{\mathds{C}}
\newcommand{\I}{\mathds{I}}
\newcommand{\N}{\mathds{N}}
\newcommand{\Q}{\mathds{Q}}
\newcommand{\R}{\mathds{R}}
\newcommand{\Z}{\mathds{Z}}

\begin{document}
\begin{enumerate}
	\item[5.3.1b] Find the remainder when 2(26!) is divided by 29.\\
		By Wilson's Theorem, 
		\[(28)! \equiv -1 \text{ (mod 29)}\]
		Then, we have
		\begin{align*}
		(28)! &\equiv -1 \text{ (mod 29)}\\
		28\cdot 27\cdot (26!) &\equiv-1 \text{ (mod 29)}\\
		-1\cdot-2\cdot(26!) &\equiv -1 \text{ (mod 29)}\\
		2(26!) &\equiv 28 \text{ (mod 29)}
		\end{align*}
		Thus,  when 2(26!) is divided by 29 the remainder is 28.
	\item[5.3.04] Show that $ 18! \equiv -1 $ (mod 437). \\
	Notice, $ 437 = 19\cdot 23 $ and $ \gcd(19,23)=1 $. Then, by Wilson's Theorem
	\[18!\equiv-1 \text{ (mod 19)}\]
	Also, by Wilson's Theorem
	\[22!\equiv-1 \text{ (mod 23)}\]
	Then, we have
	\begin{align*}
		22! &\equiv -1 \text{ (mod 23)}\\
		22\cdot 21\cdot 20\cdot 20 \cdot 18 \cdot 18! &\equiv -1 \text{ (mod 23)}\\
		24\cdot 18! &\equiv -1 \text{ (mod 23)}\\
		18! &\equiv -1 \text{ (mod 23)}
	\end{align*}
	Thus, from the two resultants, we have
	\[18!\equiv -1 \text{ (mod $ 19 \cdot 23 $)}\]
	\[18!\equiv -1 \text{ (mod 437)}\]
	Thus, $ 18! \equiv -1 $ (mod 437).
	\item[5.3.13] Supply any missing details in the following proof of the irrationality of $ \sqrt{2} $: Suppose $ \sqrt{2} = a/b $, with $ \gcd(a,b)=1 $. Then $ a^2=2b^2, $ so that $ a^2+b^2=3b^2 $. But $ 3 | (a^2+b^2) $ implies that $ 3|a $ and $ 3|b $, a contradiction.
	
	 Suppose $ \sqrt{2} = a/b $, with $ \gcd(a,b)=1 $. Then $ a^2=2b^2, $ so that $ a^2+b^2=3b^2 $.\[3 | (a^2+b^2) \Rightarrow a^2+b^2\equiv 0 \text{ (mod 3)}\]
	 Using the results from 5.3.12, if for some $k\in\Z, p=4k+3 $ is prime and $ a^2+b^2\equiv \text{ (mod p)} $, then $ a \equiv b \equiv 0 \text{ (mod p)} $\\
	 But $ 3=0\cdot k +3 $\\
	 Thus, $ a\equiv b\equiv 0 \text{ (mod 3)} $\\
	 Then,  $ 3|a $ and $ 3|b $, contradicts $ \gcd(a,b)=1 $\\
	 Therefore, $ \sqrt{2} $ is irrational.
	
	\item[5.3.17] If $ p $ and $ q $ are distinct primes, prove that for any integer a,
		\[pq|a^{pq}-a^p-a^q+a\]
		By Fermat's corollary,\\
		 $ a^p\equiv a  $ (mod $ p $). Then,\\
		 $(a^p)^q \equiv a^{pq}\equiv a^q $ (mod $ p $)\\
		 Thus, 
		 \[a^{pq}\equiv0+a^q\equiv(a^p-a)+a^q\equiv a^p+a^q-a \text{ (mod p)}\]
		 Similarly for $ q $,
		 \[a^{pq}\equiv0+a^p\equiv(a^q-a)+a^p\equiv a^p+a^q-a \text{ (mod q)}\]
		 Therefore, 
		 	\[p|a^{pq}-a^p-a^q+a \]
		 	\[q|a^{pq}-a^p-a^q+a \]
		 Thus, $pq|a^{pq}-a^p-a^q+a$
\end{enumerate}

\end{document}
