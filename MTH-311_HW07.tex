%This is a Latex file.
\documentclass[12pt]{article}
\usepackage{latexsym,fancyhdr,amsmath,amsfonts,amsthm,dsfont}
\usepackage{amssymb}

% margins are relative to the default of 1 in
%\topmargin       -0.2 in

\topmargin        -0.2 in
\textheight       8.4 in
\oddsidemargin    0 in     % this is for pages 1, 3, 5, ...
\evensidemargin   0 in     % and this for 2, 4, 6, ...
\textwidth        6.5 in
\headheight       0 in     % we won't have a running head, nor
\headsep          .35 in     % any extra space between head and text

%\parindent 0pt

\pagestyle{fancy} \lhead{\sf MTH 311} \chead{\sf Homework \#07}
\rhead{\sf Austin Klum} \lfoot{} \cfoot{} \rfoot{}

\newcommand{\C}{\mathds{C}}
\newcommand{\I}{\mathds{I}}
\newcommand{\N}{\mathds{N}}
\newcommand{\Q}{\mathds{Q}}
\newcommand{\R}{\mathds{R}}
\newcommand{\Z}{\mathds{Z}}

\begin{document}
\begin{enumerate}
	\item[3.2.03] Given that $ p \not | n $ for all primes $ p \leq \sqrt[3]{n} $, show that $ n > 1 $ is either prime or the product of two primes.
		\begin{proof}
			By way of contradiction, let us assume $ n=p_1p_2\cdots p_k , k\geq3$ The first three prime factors are $ p_1,p_2,p_3 $. Then, $ p_1 | n , p_2|n, $ and $ p_3|n \Rightarrow p_1p_2p_3|n\Rightarrow p\geq p_1p_2p_3$.
			Since $ p\not|n $ for all $ p\leq \sqrt[3]{n} $. Then, $ p_1>\sqrt[3]{n},p_2>\sqrt[3]{n} $, and $ p_3>\sqrt[3]{n} $. Thus, $ p_1p_2p_3>n $ is a contradiction.\\
			Therefore $ n>1 $ is either a prime of the product of two primes.
		\end{proof}
	\item[3.2.05] Show that any composite three-digit number must have a prime factor less than or equal to 31.
	\begin{proof}
		Let $ n $ be a composite three-digit number. Then $ n \leq 999 $. As $ n $ is composite, there must exist a $ p $ where $ p \not = n $, $ p|n $ and $ p\leq \sqrt{n}\Rightarrow p\leq\sqrt{999} \Rightarrow p\leq 31 $\\
		Therefore, any composite three-digit number must have a prime factor less than or equal to 31.
	\end{proof}
	
	\item[3.2.9a] Prove that if $ n >2 $, then there exists a prime $ p $ satisfying $ n < p <n! $.
		\begin{proof}
			Let $ n > 2 $. Using the hint given, if $ (n! - 1)  $ is prime, then the statement is satisfied. \\
			Else, $ (n! - 1) $ is composite and thus has a prime divisor $ p $ with $ p\leq n \Rightarrow p|n! \text{ and } p|(n!-1) \Rightarrow p|(n!-(n!-1))=1 \Rightarrow p > n$. But $ p > n $ is a contradiction. \\
			Therefore, if $ n >2 $, then there exists a prime $ p $ satisfying $ n < p <n! $
		\end{proof}
	\item[3.2.9b] For $ n > 1 $, show that every prime divisor of $ n! + 1 $ is an odd integer that is greater than $ n $.
	\begin{proof}
		Let $ n > 1 $. Then,
		\begin{align*}
			n! + 1 &= n(n-1)(n-2)\cdots 3 \cdot 2 \cdot 1 + 1\\
				   &= 2n(n-1)(n-2)\cdots 3 + 1\\
				   &\text{ let } m:=n(n-1)(n-2)\cdots 3 \\
				   &= 2m+1
		\end{align*}
		Thus, $ 2m+1 $ is odd and greater than $ n $.
		Therefore,  For $ n > 1 $, every prime divisor of $ n! + 1 $ is an odd integer that is greater than $ n $
	\end{proof}
\end{enumerate}

\end{document}
