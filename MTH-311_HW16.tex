%This is a Latex file.
\documentclass[12pt]{article}
\usepackage{latexsym,fancyhdr,amsmath,amsfonts,amsthm,dsfont}
\usepackage{amssymb}

% margins are relative to the default of 1 in
%\topmargin       -0.2 in

\topmargin        -0.2 in
\textheight       8.4 in
\oddsidemargin    0 in     % this is for pages 1, 3, 5, ...
\evensidemargin   0 in     % and this for 2, 4, 6, ...
\textwidth        6.5 in
\headheight       0 in     % we won't have a running head, nor
\headsep          .35 in     % any extra space between head and text

%\parindent 0pt

\pagestyle{fancy} \lhead{\sf MTH 311} \chead{\sf Homework \#16}
\rhead{\sf Austin Klum} \lfoot{} \cfoot{} \rfoot{}

\newcommand{\C}{\mathds{C}}
\newcommand{\I}{\mathds{I}}
\newcommand{\N}{\mathds{N}}
\newcommand{\Q}{\mathds{Q}}
\newcommand{\R}{\mathds{R}}
\newcommand{\Z}{\mathds{Z}}

\begin{document}
\begin{enumerate}
	\item[7.3.02] Use Euler's theorem to confirm that, for any integer $ n \geq 0 $,
	\[51|10^{32n+9}-7\]
	Since, $ 51 = 17\cdot3$ we have $ \phi(51)=\phi(17)\phi(3)=16\cdot2=32 $. By Euler's theorem we have $ 10^{32} \equiv 1 $ (mod 51). Notice,
		\[10^{32n+9}-7 = 10^{32n}\cdot10^9-7 = 1^n \cdot 10^9-7=10^9-7\equiv0 \text{ (mod 51)}\]
	Then observe the following. 
	\begin{align*}
		10^9=(10^3)^3 &= 1000^3 \\
			&\equiv 31^3 \text{ (mod 51)}\\
			&\equiv 29791 \text{ (mod 51)}\\
	    	&\equiv 7 \text{ (mod 51)}
	\end{align*}
	Thus, we have $ 10^9 \equiv 7 $ (mod 51).\\
	Therefore, $51|10^{32n+9}-7$
		
	\item[7.3.05] If $ m $ and $ n $ are relatively prime positive integers, prove that
	\[m^{\phi(n)}+n^{\phi(m)}\equiv 1 \text{ (mod $mn  $)} \]
		By Euler's Theorem, we have $ m^{\phi(n)}\equiv1 $ (mod $ n $) and $ n^{\phi(m)}\equiv 1 $ (mod $ m $). Also, by Euler's Theorem we get $ m^{\phi(n)} \equiv 0$ (mod $ m $) and $ n^{\phi(m)}\equiv0 $ (mod $ n $). Notice,
		\[m^{\phi(n)}+n^{\phi(m)}\equiv(1+0)\equiv 1 \text{ (mod $ m $)} \]
		and
		\[m^{\phi(n)}+n^{\phi(m)}\equiv(1+0)\equiv 1 \text{ (mod $ m $)} \]
		Therefore, $m^{\phi(n)}+n^{\phi(m)}\equiv 1 \text{ (mod $mn  $)}$
	
	\item[7.3.8a] If $ \gcd(a,n)=1 $, show that the linear congruence $ ax \equiv b$ (mod $ n $) has the solution $ x \equiv ba^{\phi(n)-1} $ (mod $ n $).\\
	Let $ \gcd(a,n)=1 $. Then by Euler's Theorem we have $ a^{\phi(n)} \equiv 1 $ (mod $n$). Observe. 
	\begin{align*}
		 &a^{\phi(n)}  \equiv 1 \text{ (mod $n$)}\\
		 &\Rightarrow ba^{\phi(n)} \equiv b \text{ (mod $n$)}\\
		 &\Rightarrow aa^{\phi(n)-1}b \equiv b \text{ (mod $n$)}\\
		 &\Rightarrow a(^{\phi(n)-1}b) \equiv b \text{ (mod $n$)}\\
	\end{align*}
	Thus, the linear congruence $ ax\equiv b $ (mod n) has the solution $ x=ba^{\phi(n)-1} $ (mod n)
	\item[7.3.10] For any integer $ a $, show that $ a $ and $ a^{4n+1} $ have the same last digit.
	\\
	If $ \gcd(a,10)=1 $, then $ a^{\phi(10)}\equiv1 $ (mod 10). Thus, as $ \phi(10) = 4$, we have $ a^4\equiv 1 $ (mod 10). Then we have $ a^{4n}\equiv 1 $ (mod 10) and $ a^{4n+1}\equiv a $ (mod 10) \\
	Notice, then
	 \[a^{4n+1}\equiv a\text{ (mod 2)}\]
	 	is true , because it is obviously true for $ a\equiv0 $ (mod 2) and $ a\equiv 1 $ (mod 2)
	\\
	Suppose $ \gcd(a,5) = 1$. Then by Fermat's Theorem, $ a^4\equiv 1 $ (mod 5). Then $ a^{4n+1}=(a^4)^n\cdot a \equiv a $ (mod 5). Thus, $ a^{4n+1}\equiv a $ (mod 5) \\
	\\
	Therefore, for any integer $ a $, $ a $ and $ a^{4n+1} $ have the same last digit
\end{enumerate}
\end{document}
