%This is a Latex file.
\documentclass[12pt]{article}
\usepackage{latexsym,fancyhdr,amsmath,amsfonts,amsthm,dsfont}
\usepackage{amssymb}

% margins are relative to the default of 1 in
%\topmargin       -0.2 in

\topmargin        -0.2 in
\textheight       8.4 in
\oddsidemargin    0 in     % this is for pages 1, 3, 5, ...
\evensidemargin   0 in     % and this for 2, 4, 6, ...
\textwidth        6.5 in
\headheight       0 in     % we won't have a running head, nor
\headsep          .35 in     % any extra space between head and text

%\parindent 0pt

\pagestyle{fancy} \lhead{\sf MTH 311} \chead{\sf Homework \#18}
\rhead{\sf Austin Klum} \lfoot{} \cfoot{} \rfoot{}

\newcommand{\C}{\mathds{C}}
\newcommand{\I}{\mathds{I}}
\newcommand{\N}{\mathds{N}}
\newcommand{\Q}{\mathds{Q}}
\newcommand{\R}{\mathds{R}}
\newcommand{\Z}{\mathds{Z}}

\begin{document}
\begin{enumerate}
	\item[8.2.1b]If $ p $ is an odd prime, then the congruence $ x^{p-2}+\cdots+x^2+x+1\equiv0 $ (mod $ p $) has exactly $ p-2 $ incongruent solutions, and they are the integers $ 2,3,\cdots,p-1 $
	\begin{proof}
		Notice, as $ p $ is an odd prime and for $ 1 \leq x \leq p-1,$ we have $ \gcd(x,p)=1 $. Then by Fermat's Theorem we have $ x^{p-1}\equiv 1 $ (mod $ p $) $ \Rightarrow x^{p-1}-1\equiv0$ (mod $ p $) has exactly $ p-1 $ solutions. Notice, 
			\[x^{p-1}=(x-1)(x^{p-2}+x^{p-3}+\cdots+x^2+x+1)\]
		Then $ x-1\equiv0 $ (mod $ p $) has exactly 1 solution. Also, we then have $ x^{p-2}+\cdots+x+1 $ has exactly $ (p-1)-1=p-2 $ solutions. As $ x\not\equiv1 $ (mod $ p $) for $ 2 \leq x \leq p-1  $ and $ x^{p-1}-1\equiv0 $ for $ 2 \leq x \leq p-1 $. Then, $ x^{p-2}+\cdots+x+1\equiv0 $\\
		Therefore, the congruence $ x^{p-2}+\cdots+x^2+x+1\equiv0 $ (mod $ p $) has exactly $ p-2 $ incongruent solutions, and they are the integers $ 2,3,\cdots,p-1$
	\end{proof}
	
	\item[8.2.2b] Verify the congruence $ x^2\equiv-1 $ (mod 65) has four incongruent solutions; hence, Lagrange's theorem need not hold if the modulus is a composite number.\\
		Note, $ 65 = 5\cdot13 $. Then
		\begin{align*}
			x^2 &\equiv -1 \text{ (mod $5$)} & x^2 &\equiv -1 \text{ (mod $ 13 $)} \\
			x^2 &\equiv 4 \text{ (mod $5$)} & x^2 &\equiv 12, x^2 \equiv 25 \text{ (mod $ 13 $)} \\
			(x+2)(x-2)&\equiv 0\text{ (mod $5$)} & (x+5)(x-5)&\equiv 0 \text{ (mod $13$)} 
		\end{align*}
		Thus, $ x \equiv 8,18,47,57 $ (mod 65)
	\item[8.2.3b] Determine the roots of the prime $ p = 19 $ expressing $p$ as a power of some one of the roots.\\
	Notice, $ \phi(p-1)=\phi(18)=6 $. Then we have $ x^{18} \equiv 1 $ (mod 19). We then have $ x=2 $ being a primitive root. Determining the rest we have $ 2^k $ where $ k\in\Z $ and $\gcd(k,18)=1 $. Thus, $ k =1,5,7,11,13,17 $\\
	Therefore, the primitive roots are $ 2^1, 2^5,2^7,2^{11},2^{13},2^{17} $
	
	\item[8.2.6a] Assuming that $ r $ is primitive root of the odd prime $ p $, establish: The congruence $ r^{(p-1)/2}\equiv-1 $ (mod $ p $) holds.
	\begin{proof}
		Note, that by Fermat's Theorem we have $ r^{p-1} \equiv 1 \pmod{p} $. Then, \[ p|r^{p-1}-1=(r^{(p-1)/2}-1)(r^{(p-1)/2}+1)\]
		Notice, that $ p\not| r^{(p-1)/2}-1 $, as then $ r^{(p-1)/2}\equiv 1 \pmod p $, which implies that $ r $ is not a primitive root. \\
		Therefore, we must have $ r^{(p-1)/2} \equiv -1 \pmod p $.\\
	\end{proof}
\end{enumerate}
\end{document}
