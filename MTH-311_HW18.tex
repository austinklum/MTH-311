%This is a Latex file.
\documentclass[12pt]{article}
\usepackage{latexsym,fancyhdr,amsmath,amsfonts,amsthm,dsfont}
\usepackage{amssymb}

% margins are relative to the default of 1 in
%\topmargin       -0.2 in

\topmargin        -0.2 in
\textheight       8.4 in
\oddsidemargin    0 in     % this is for pages 1, 3, 5, ...
\evensidemargin   0 in     % and this for 2, 4, 6, ...
\textwidth        6.5 in
\headheight       0 in     % we won't have a running head, nor
\headsep          .35 in     % any extra space between head and text

%\parindent 0pt

\pagestyle{fancy} \lhead{\sf MTH 311} \chead{\sf Homework \#17}
\rhead{\sf Austin Klum} \lfoot{} \cfoot{} \rfoot{}

\newcommand{\C}{\mathds{C}}
\newcommand{\I}{\mathds{I}}
\newcommand{\N}{\mathds{N}}
\newcommand{\Q}{\mathds{Q}}
\newcommand{\R}{\mathds{R}}
\newcommand{\Z}{\mathds{Z}}

\begin{document}
\begin{enumerate}
	\item[8.2.1b]If $ p $ is an odd prime, then the congruence $ x^{p-2}+\cdots+x^2+x+1\equiv0 $ (mod $ p $) has exactly $ p-2 $ incongruent solutions, and they are the integers $ 2,3,\cdots,p-1 $
	
	\item[8.2.2b] Verify the congruence $ x^2\equiv-1 $ (mod 65) has four incongruent solutions; hence, Lagrange's theorem need not hold if the modulus is a composite number.
	
	\item[8.2.3b] Determine the roots of the prime $ p = 19 $ expressing $p$ as a power of some one of the roots.
	
	\item[8.2.6a] Assuming that $ r $ is primitive root of the odd prime $ p $, establish: The congruence $ r^{(p-1)/2}\equiv-1 $ (mod $ p $) holds.

\end{enumerate}
\end{document}
