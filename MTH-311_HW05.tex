%This is a Latex file.
\documentclass[12pt]{article}
\usepackage{latexsym,fancyhdr,amsmath,amsfonts,amsthm,dsfont}
\usepackage{amssymb}

% margins are relative to the default of 1 in
%\topmargin       -0.2 in

\topmargin        -0.2 in
\textheight       8.4 in
\oddsidemargin    0 in     % this is for pages 1, 3, 5, ...
\evensidemargin   0 in     % and this for 2, 4, 6, ...
\textwidth        6.5 in
\headheight       0 in     % we won't have a running head, nor
\headsep          .35 in     % any extra space between head and text

%\parindent 0pt

\pagestyle{fancy} \lhead{\sf MTH 311} \chead{\sf Homework \#04}
\rhead{\sf Austin Klum} \lfoot{} \cfoot{} \rfoot{}

\newcommand{\C}{\mathds{C}}
\newcommand{\I}{\mathds{I}}
\newcommand{\N}{\mathds{N}}
\newcommand{\Q}{\mathds{Q}}
\newcommand{\R}{\mathds{R}}
\newcommand{\Z}{\mathds{Z}}

\begin{document}
\begin{enumerate}
	\item[2.4.2b] Use the Euclidean Algorithm to obtain integers $ x $ and $ y $ that satisfy $\gcd(24,138)=24x+138y$\\
	Using the Euclidean Algorithm, observe.
		\begin{align*}
			138 &= 5(24)+18\\
			24 &= 1(18)+6\\
			18 &= 3(6)+0
		\end{align*}
		Working these results in reverse, notice.
		\begin{align*}
			6 &= 24 - 18(1)\\
			  &= 24 - 24(138-15)\\
			  &= 6(24)-1(138)
		\end{align*}
		Thus, $ x=6 $ and $ y= -1 $.
	\item[2.4.2c] Use the Euclidean Algorithm to obtain integers $ x $ and $ y $ that satisfy $\gcd(119,272)=119x+272y$\\	
		Using the Euclidean Algorithm, observe.
		\begin{align*}
			272 &= 2(119)+34\\
			119 &= 3(34)+17\\
			34 &= 2(17)+0
		\end{align*}
		Working these results in reverse, notice.
		\begin{align*}
			 17	&= 119 - 3(34)\\
				&= 119 - 3(272-2(119))\\
				&= 7(119)-3(272)
		\end{align*}
	Thus, $ x=7 $ and $ y= -3 $.
	\item[2.4.4c] Assuming that $ \gcd(a,b)=1 $, prove the following: $ \gcd(a+b,a^2+b^2)=1 $ or 2.\\
		\begin{proof}
			Let $ d=\gcd(a,b)=1 $. Then, $ \gcd(a,b)=1$ can be written as $ax+by=1$.
				\[(2a)x+(2b)y=2\]
			Then, $ d|2 \Rightarrow d = 1$ or 2.
			Therefore,  $ \gcd(a+b,a^2+b^2)=1 $ or 2.
		\end{proof}
	\item[2.4.6] Prove that if $ \gcd(a,b)=1, $ then $ \gcd(a+b,ab)=1 $
		\begin{proof}
			Let $ \gcd(a,b)=1 $. Then, $ \gcd(a,b)=1$ can be written as $ax+by=1$. Squaring both sides yields in \[a^2x^2+2abxy+b^2y^2=1\].
			Notice. 
				\[a^2x^2+2abxy+b^2y^2=ab(2xy-x^2-y^2)+(a+b)(ax^2+by^2)\]
				This can be rewritten as,
				\[ab(x)+(a+b)(y)\]
				Therefore, $ \gcd(a+b,ab)=1 $
		\end{proof}
        
\end{enumerate}

\end{document}
