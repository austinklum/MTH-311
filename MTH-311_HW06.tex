%This is a Latex file.
\documentclass[12pt]{article}
\usepackage{latexsym,fancyhdr,amsmath,amsfonts,amsthm,dsfont}
\usepackage{amssymb}

% margins are relative to the default of 1 in
%\topmargin       -0.2 in

\topmargin        -0.2 in
\textheight       8.4 in
\oddsidemargin    0 in     % this is for pages 1, 3, 5, ...
\evensidemargin   0 in     % and this for 2, 4, 6, ...
\textwidth        6.5 in
\headheight       0 in     % we won't have a running head, nor
\headsep          .35 in     % any extra space between head and text

%\parindent 0pt

\pagestyle{fancy} \lhead{\sf MTH 311} \chead{\sf Homework \#05}
\rhead{\sf Austin Klum} \lfoot{} \cfoot{} \rfoot{}

\newcommand{\C}{\mathds{C}}
\newcommand{\I}{\mathds{I}}
\newcommand{\N}{\mathds{N}}
\newcommand{\Q}{\mathds{Q}}
\newcommand{\R}{\mathds{R}}
\newcommand{\Z}{\mathds{Z}}

\begin{document}
\begin{enumerate}
	\item[3.1.3a] Prove the assertion, any prime of the form $ 3n+1 $ is also of the form $ 6m+1 $
		\begin{proof}
			
		Let $ p $ be a prime of the form $ 3n+1 $. Assume $ n $ is odd. Then $ 3n $ is odd, $ 3n+1 $ is even. Also $ 3n+1 > 2 , $ for all $ n \geq 1 $. Thus, $ p=3n+1 $ cannot be prime if n is odd. Thus, $ n $ must be even. Let $ n=2k $ for some $ k\in\Z^+ $. Notice.
		\[p=3n+1=3(2k)+1=6k+1\]
		Therefore, any prime of the form $ 3n+1 $ is also of the form $ 6m+1 $
	\end{proof}
	\item[3.1.3e] Prove the assertion, the only prime of the form $ n^2-4 $ is 5.
		\begin{proof}
			$ 5= 3^2-4 $ so, 5 is of the form $ n^2-4 $.\\
			For $ n >5 ,n^2-4>0$. Since $ n>3\Rightarrow n^2-4=(n-2)(n+2) $. \\
			Then, $ (n-2) > 1 $ and it's a factor of $ n^2-4 $, so $ n^2-4 $ cannot be prime.
			Therefore, the only prime of the form $ n^2-4 $ is 5.
		\end{proof}
	\item[3.1.04] If $ p\geq 5 $ is a prime number, show that $ p^2+2 $ is composite.
		\begin{proof}
			Let $ p \geq 5 $ be a prime number. Then,
			\[p^2+2=(p^2-1)+3=(p-1)(p+1)+3\]
			But $ p $ on the form $ p=3k+1 $ or $ p=3k+2 $
			\begin{enumerate}
				\item[Case 1:] $ p=3k+1 \Rightarrow p-1=3k\Rightarrow 3 | (p-1)\Rightarrow 3|(p-1)(p+1)+3 $. So, $ 3|(p^2+2)\Rightarrow p^2+2 $ is composite.
				\item[Case 2:] $ p=3k+2\Rightarrow p+1 = 3(k+1)\Rightarrow 3|(p+1)\Rightarrow 3|(p-1)(p+1)+3 $. So, $ 3|(p^2+2)\Rightarrow p^2+2 $ is composite.
			\end{enumerate}
		Therefore, for $ p\geq 5 $ that is a prime number, $ p^2+2 $ is composite.
		\end{proof}
	\item[3.1.10] If $ p \not = 5 $ is an odd prime, prove that either $ p^2-1 $ or $ p^2+1 $ is divisible by 10.
	\begin{proof}
		Let $ p \not = 5 $ be an odd prime. Then by division algorithm $ p=10q+r $, for some $ q,r\in\Z.$ $ r $ must be $1,3,7,$ or 9.
		\begin{enumerate}
			\item [Case 1:] $ r = 1, p=10q+ 1 \Rightarrow p^2 - 1 =100q^2+20q=10(10q^2+2q+0)\Rightarrow 10 | (p^2-1) $.
			\item [Case 2:] $ r = 3, p=10q+ 3 \Rightarrow p^2 + 1 =100q^2+60q+10=10(10q^2+6q+1)\Rightarrow 10 | (p^2+1) $.
			\item [Case 3:] $ r = 7, p=10q+ 7 \Rightarrow p^2 + 1 =100q^2+140q+50=10(10q^2+14q+5)\Rightarrow 10 | (p^2+1) $.
			\item [Case 4:] $ r = 9, p=10q+ 9 \Rightarrow p^2 - 1 =100q^2+180q+80=10(10q^2+18q+8)\Rightarrow 10 | (p^2-1) $.
		\end{enumerate}
		Therefore, in all cases either $ 10|(p^2-1) $ or $ 10|(p^2+1) $
	\end{proof}
	\item[3.1.15] Prove that a positive integer $ a > 1 $ is a square if and only if in the canonical form of $ a $ all the exponents of the primes are even integers.
	\begin{proof}
		Let $ a $ be a square, $ a=k^2 , k=p_1^{r_1}p_2^{r_2}\cdots p_s^{r_s} , p_i$ is prime.\\ Then, $ a=k^2=(p_1^{r_1}p_2^{r_2}\cdots p_s^{r_s})^2=p_1^{2r_1}p_2^{2r_2}\cdots p_s^{2r_s} $\\
		Thus, all the exponents are even.
		\\
		Assume all the exponents are even. That is $ a = p_1^{2m_1}p_2^{2m_2}\cdots p_s^{2m_s}$\\
		$ a=(p_1^{m_1}p_2^{m_2}\cdots p_s^{m_s})^2 $\\
		Thus, $ a $ is square.\\
		Therefore, a positive integer $ a > 1 $ is a square if and only if in the canonical form of $ a $ all the exponents of the primes are even integers.
	\end{proof}
\end{enumerate}

\end{document}
