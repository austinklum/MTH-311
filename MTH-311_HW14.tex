%This is a Latex file.
\documentclass[12pt]{article}
\usepackage{latexsym,fancyhdr,amsmath,amsfonts,amsthm,dsfont}
\usepackage{amssymb}

% margins are relative to the default of 1 in
%\topmargin       -0.2 in

\topmargin        -0.2 in
\textheight       8.4 in
\oddsidemargin    0 in     % this is for pages 1, 3, 5, ...
\evensidemargin   0 in     % and this for 2, 4, 6, ...
\textwidth        6.5 in
\headheight       0 in     % we won't have a running head, nor
\headsep          .35 in     % any extra space between head and text

%\parindent 0pt

\pagestyle{fancy} \lhead{\sf MTH 311} \chead{\sf Homework \#14}
\rhead{\sf Austin Klum} \lfoot{} \cfoot{} \rfoot{}

\newcommand{\C}{\mathds{C}}
\newcommand{\I}{\mathds{I}}
\newcommand{\N}{\mathds{N}}
\newcommand{\Q}{\mathds{Q}}
\newcommand{\R}{\mathds{R}}
\newcommand{\Z}{\mathds{Z}}

\begin{document}
\begin{enumerate}
	\item[6.1.06] For any integer $ n \geq 1 $, establish the inequality $ \tau(n)\leq2\sqrt{n} $
	\begin{proof}
		Assume $ d | n $. Then, $ d $ or $ n/d \leq 2\sqrt{n}$. Note, there can be at most $ \sqrt{n} $ divisor pairs $ (d,n/d) $. \\
		Thus, $ \sqrt{2} \leq 2\sqrt{2} $.\\
		Therefore, $\tau(n)\leq2\sqrt{n} $
	\end{proof}
	\item[6.1.08] Show that $ \sum_{d|n} 1/d =\sigma(n)/n $ for every positive integer $ n $
	\begin{proof}
		Note, $ d $ is a divisor of $ n \Rightarrow n/d $ is a divisor of $ n $, as $ d\cdot \frac{n}{d}=n $. Let the set of divisors of $ n = \{d_1,d_2,\cdots,d_k\} \Rightarrow \{n/d_1,n/d_2,\cdots,n/d_k\} $ Then,
		\[\tau(n)=d_1+d_2+\cdots+d_k=\frac{n}{d_1}+\frac{n}{d_2}+\cdots+\frac{n}{d_k}=n(\frac{1}{d_1}+\frac{1}{d_2}+\cdots+\frac{1}{d_k})\]
		Thus, $ \frac{\tau(n)}{n}=\frac{1}{d_1}+\frac{1}{d_2}+\cdots+\frac{1}{d_k} = \sum_{d|n} 1/d $\\
		Therefore, $ \sum_{d|n} 1/d =\sigma(n)/n $ for every positive integer $ n $
	\end{proof}
	
	\item[6.1.14a] For $ k \geq 2 $, $ n=2^{k-1} $ satisfies the equation $ \sigma(n)=2n-1 $
	\begin{proof}
	Notice,
	\[\sigma(n)=\sigma(2^{k-1})=\frac{2^{k-1+1}-1}{2-1}=2^k-1=2\cdot2^{k-1}-1=2n-1\]
	\end{proof}
	\item[6.1.14b] For $ k \geq 2 $, if $ 2^k-1 $ is prime, then $n=2^{k-1}(2^k-1)  $ satisfies the equation $ \sigma(n) = 2n $
	 \begin{proof}
	 Assume $ 2^k-1 $ is prime. Notice, 
	 \begin{align*}
	 \sigma(n)&=\frac{2^{k-1+1}-1}{2-1}\cdot\frac{(2^k-1)^2-1}{2^k-1-1}\\
	 		  &=(2^k-1)(2^k-1+1)=(2^k-1)(2^k)\\
	 		  &= 2(2^{k-1})(2^{k-1})\\
	 		  &= 2n
	 \end{align*}
	 	Therefore, $n=2^{k-1}(2^k-1)  $ satisfies the equation $ \sigma(n) = 2n $
	 \end{proof}
	\item[6.1.14c] For $ k \geq 2 $, if $ 2^k-3 $ is prime, then $n= 2^{k-1}(2^k-3) $ satisfies $ \sigma(n) = 2n+2 $
		 \begin{proof}
		Assume $ 2^k-3 $ is prime. Notice, 
		\begin{align*}
		\sigma(n)&=\frac{2^{k-1+1}-1}{2-1}\cdot\frac{(2^k-3)^2-1}{2^k-3-1}\\
		&=(2^k-1)(2^k-3+1)=(2^k-1)(2^k-3+1)\\
		&= (2^k-1)(2^k-2)\\
		&= 2^{2k}-3\cdot 2^k+2\\
		&= 2^k(2^k-3)+2\\
		&=2(2^{k-1})(2^k-3)+2\\
		&= 2n+2
		\end{align*}
		Therefore,  $n= 2^{k-1}(2^k-3) $ satisfies $ \sigma(n) = 2n+2 $
	\end{proof}
\end{enumerate}

\end{document}
