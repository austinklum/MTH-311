%This is a Latex file.
\documentclass[12pt]{article}
\usepackage{latexsym,fancyhdr,amsmath,amsfonts,amsthm,dsfont}
\usepackage{amssymb}

% margins are relative to the default of 1 in
%\topmargin       -0.2 in

\topmargin        -0.2 in
\textheight       8.4 in
\oddsidemargin    0 in     % this is for pages 1, 3, 5, ...
\evensidemargin   0 in     % and this for 2, 4, 6, ...
\textwidth        6.5 in
\headheight       0 in     % we won't have a running head, nor
\headsep          .35 in     % any extra space between head and text

%\parindent 0pt

\pagestyle{fancy} \lhead{\sf MTH 311} \chead{\sf Homework \#09}
\rhead{\sf Austin Klum} \lfoot{} \cfoot{} \rfoot{}

\newcommand{\C}{\mathds{C}}
\newcommand{\I}{\mathds{I}}
\newcommand{\N}{\mathds{N}}
\newcommand{\Q}{\mathds{Q}}
\newcommand{\R}{\mathds{R}}
\newcommand{\Z}{\mathds{Z}}

\begin{document}
\begin{enumerate}
	\item[4.2.03] If $ a \equiv b $ (mod $ n ) $ prove that $ \gcd(a,n) = \gcd(b,n) $.
	\begin{proof}
		Let $ a \equiv b $ (mod $ n ) $ then $ n | a-b \Rightarrow a-b = nk $ for some $ k\in\Z $. Let $ d = \gcd(a,n) $ and $ e = \gcd(b,n) $. Then, $ d | a $ and $ d|n \Rightarrow d |(a-nk) \Rightarrow d|b$. Using this fact, as $ d | n $ and $ d | b  \Rightarrow d | \gcd(b,n) \Rightarrow d | e $ \\
		Going the other direction, $ e | b $ and $ e | n \Rightarrow e|(b+nk)\Rightarrow e|a$.\\
		Thus, $ e|n $ and $ e|a \Rightarrow g|\gcd(a,n) \Rightarrow e|d $\\
		Therefore, as $ e|d $ and $ d|e $, we have $ \gcd(a,n) = \gcd(b,n) $ 
	\end{proof}
	\item[4.2.6c] For $ n \geq, $ use congruence theory to establish each of the following divisibility statement: $ 27 | 2^{5n+1} + 5^{n+2} $
	\begin{proof}
		Notice that $ 32 \equiv 5 ($ mod 27). Thus, $ 2^5 \equiv $ (mod 27)\\
		Then, we have $ 2^{5n} \equiv 5^n $ (mod 27). Then $ 2^{5n}\cdot 2 \equiv 2 \cdot 5^n $(mod 27)\\
		Observe.
		\begin{align*}
			2^{5n+1}+5^{n+2} &\equiv 2\cdot 5^n + 5^{n+2} \text{ (mod 27)}\\
			                 &\equiv 5^n (2 + 25) \text{ (mod 27)}\\
			                 &\equiv 5^n \cdot 27 \text{ (mod 27)}\\
			                 &\equiv 0 \text{ (mod 27)}
		\end{align*}
		Therefore, $ 27 | 2^{5n+1} + 5^{n+2} $
	\end{proof}
	\item[4.2.8d] Prove if the integer $ a $ is not divisible by 2 or 3, then $ a^2 \equiv 1 $ (mod $ 24 ) $ .
		\begin{proof}
			Let $ a\in \Z $ such that $ a $ is not divisible by 2 or 3. As $ a $ is not divisible by 2, then $ a $ is odd. Notice. For some $ k\in\Z. $
			\[a^2=(2k+1)^2=2k^2+4k+1=4k(k+1)+1\]
			By looking at the parity, we know that $ 2|K(k+1)\Rightarrow k(k+1)=2l $ for some $ l \in \Z $.
			Thus, $ 4(2l)+1=8l+1 $.\\
			Thus, $ a^2 \equiv 1$ (mod 8). Then, $ 8|a^2-1 $. As $ a $ is not divisible by 3, then for some $ q\in\Z, a=3q+1 $ or $ a=3q+2 $
			\begin{enumerate}
				\item[Case $ a= 3q+1 $] $ a^2 = (3q+1)^2=9q^2+6q+1=3(3q^2+2q)+1 $\\
					So, $ a^2-1=3(3q^2+2q) \Rightarrow 3 | a^2-1$
				\item[Case $ a = 3q+2$] $ a^2 = (3q+2)^2 = 9q^2+12q+4=3(3q^2+4q+1)+1 $\\
					So, $ a^2-1 = 3(3q^2+4q+1)+1 \Rightarrow 3 | a^2-1 $
			\end{enumerate}
			Thus, in both cases $ 3 | a^2-1 $\\
			Therefore, as $ 8|a^2-1 $, $ 3|a^2-a $, and $ \gcd(3,8)=1 , \text{ then } 24|a^2-1 \Rightarrow a^2 \equiv $(mod 24)
		\end{proof}
	\item[4.2.16] Use the theory of congruences to verify that $ 89 | 2^{44} -1 $ and $ 97|2^{48}-1 $.
	\begin{align*}
		2^{44}-1 &\equiv (2^{11})^4-1 \text{ (mod 89)}\\
				 &\equiv (1)^4-1 \text { (mod 89)}\\
				 &\equiv 1-1 \text{ (mod 89)}\\
				 &\equiv 0 \text{ (mod 89)}
	\end{align*}
	Thus, $ 89|2^{44}-1 $\\
	\begin{align*}
		2^{48}-1=(2^6)^8-1 &\equiv 64^8 -1 \text{ (mod 97)}\\
		64^8=(64^2)^4 &\equiv 1^4-1 \text { (mod 97)}\\
					  &\equiv 1-1 \text { (mod 97)}\\
					  &\equiv 0 \text{ (mod 97)}
	\end{align*}
	Thus, $97|2^{48}-1 $
	\item[4.2.18] If $ a \equiv b $ (mod $ n_1 ) $  and $ a \equiv c $ (mod $ n_2 ) $ , prove that $ b \equiv c $ (mod $ n ) $  where the integer $ n = \gcd(n_1,n_2) $.
	\begin{proof}
		Let  $ a \equiv b $ (mod $ n_1 ) $  and $ a \equiv c $ (mod $ n_2 ) $. \\
		So $ n_1 | a-b \Rightarrow a-b=n_1k_1, k_1\in\Z$ \\
		and $ n_2 | a-c \Rightarrow a-c=n_2k_2, k_2\in\Z$ \\
		Thus, $ b-c = n_2k_2-n_1k_1 $.\\
		Let $ n = \gcd(n_1,n_2)\Rightarrow n|n_1$ and $ n|n_2 $.\\
		So $ n|(n_2k_2-n_1k_1)\Rightarrow n|b-c $\\
		Thus,  $ b \equiv c $ (mod $ n ) $ where $ n = \gcd(n_1,n_2) $
	\end{proof}
\end{enumerate}

\end{document}
