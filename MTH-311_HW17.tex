%This is a Latex file.
\documentclass[12pt]{article}
\usepackage{latexsym,fancyhdr,amsmath,amsfonts,amsthm,dsfont}
\usepackage{amssymb}

% margins are relative to the default of 1 in
%\topmargin       -0.2 in

\topmargin        -0.2 in
\textheight       8.4 in
\oddsidemargin    0 in     % this is for pages 1, 3, 5, ...
\evensidemargin   0 in     % and this for 2, 4, 6, ...
\textwidth        6.5 in
\headheight       0 in     % we won't have a running head, nor
\headsep          .35 in     % any extra space between head and text

%\parindent 0pt

\pagestyle{fancy} \lhead{\sf MTH 311} \chead{\sf Homework \#17}
\rhead{\sf Austin Klum} \lfoot{} \cfoot{} \rfoot{}

\newcommand{\C}{\mathds{C}}
\newcommand{\I}{\mathds{I}}
\newcommand{\N}{\mathds{N}}
\newcommand{\Q}{\mathds{Q}}
\newcommand{\R}{\mathds{R}}
\newcommand{\Z}{\mathds{Z}}

\begin{document}
\begin{enumerate}
	\item[8.1.2b] If $ a $ has order $ 2k $ modulo the odd prime $ p $, then $ a^k \equiv -1 $ (mod $ p $)
		\begin{proof}
			Let $ p $ be an odd prime and $ a^{2k} \equiv 1$ (mod p). Notice, 
			\[(a^k)^2-1 \equiv 0 \text{ (mod $ p $)}\Rightarrow (a^k-1)(a^k+1)\equiv 0 \text{ (mod p)}\]
			Then, $ p | (a^k-1)(a^k+1) \Rightarrow p|(a^k+1)$. So, $ a^k\equiv -1 $ (mod $ p $)\\
			Therefore, If $ a $ has order $ 2k $ modulo the odd prime $ p $, then $ a^k \equiv -1 $ (mod $ p $)
		\end{proof}
	\item[8.1.8a] Prove that if $ p $ and $ q $ are odd primes and $ q|a^p-1 $, then either $ q|a-1 $ or else $ q=2kp+1 $ for some integer $ k $.
	\begin{proof}
		Let $ p $ and $ q $ be odd primes and $ q|a^p-1 $ Note, $ \gcd(a,q)=1 $ and $ a^p\equiv 1 $ (mod $ q $). Let $ r $ be the order of $ a $ modulo $ q $. Then, $ r|p $. As $ p $ is prime, we have $ r=1 $ or $ r=p $.
		If $ r =1 $, we have $ a\equiv 1 $ (mod $ q $) $ \Rightarrow q|(a-1) $\\
		If $ r=p $, we have $ a^{\phi(q)} \equiv 1 $ (mod $ q $). Then, $ p|\phi(q) \Rightarrow p|q-1 $. There must be some $ m $ such that $ pm=q-1 $. As $ q $ is odd, $ q-1 $ must be even, and as $ p $ is odd, $ m $ must be even, so $ m=2k $ for some $ k\in\Z $.\\
		Thus, $ p(2k)=q-1 \Rightarrow q=2pk+1 $. \\
		Therefore,  if $ p $ and $ q $ are odd primes and $ q|a^p-1 $, then either $ q|a-1 $ or else $ q=2kp+1 $ for some integer $ k $
	\end{proof} 
	\item[8.1.10] Let $ r $ be a primitive root of the integer $ n $. Prove that $ r^k $ is a primitive root of $ n $ if and only if $ \gcd(k,\phi(n))= 1 $.
	\begin{proof}
		As $ r $ has order $ \phi(n) $ (mod $ n $), we then have $ r^k $ has order $ \phi(n)/ \gcd(k,\phi(n)) $. \\
		\\
		Assume $ \gcd(k,\phi(n))=1 $ then $ r^k $ has order $ \phi(n) $.\\
		Thus, $ r^k $ is a primitive root of $ n $\\
		\\
		Suppose $ r^k $ is a primitive root of $ n $. Then $ r^k $ has order $ \phi(n) $.As $ \phi(n) $ is $ \phi(n)/\gcd(k,\phi(n))$. \\
		Thus, $ \gcd(k,\phi(n))=1 $.\\
		\\
		Therefore, $ r^k $ is a primitive root of $ n $ if and only if $ \gcd(k,\phi(n))= 1 $.
	\end{proof}
\end{enumerate}
\end{document}
