%This is a Latex file.
\documentclass[12pt]{article}
\usepackage{latexsym,fancyhdr,amsmath,amsfonts,amsthm,dsfont}
\usepackage{amssymb}

% margins are relative to the default of 1 in
%\topmargin       -0.2 in

\topmargin        -0.2 in
\textheight       8.4 in
\oddsidemargin    0 in     % this is for pages 1, 3, 5, ...
\evensidemargin   0 in     % and this for 2, 4, 6, ...
\textwidth        6.5 in
\headheight       0 in     % we won't have a running head, nor
\headsep          .35 in     % any extra space between head and text

%\parindent 0pt

\pagestyle{fancy} \lhead{\sf MTH 311} \chead{\sf Homework \#08}
\rhead{\sf Austin Klum} \lfoot{} \cfoot{} \rfoot{}

\newcommand{\C}{\mathds{C}}
\newcommand{\I}{\mathds{I}}
\newcommand{\N}{\mathds{N}}
\newcommand{\Q}{\mathds{Q}}
\newcommand{\R}{\mathds{R}}
\newcommand{\Z}{\mathds{Z}}

\begin{document}
\begin{enumerate}
	\item[3.3.02a] If 1 is added to a product of twin primes, prove that a perfect square is always obtained.
		\begin{proof}
			Let $ p $ and $ p+2 $ be twin primes. Then,
					\[p(p+2)+1=p^2+2p+1=(p+1)^2\]
			Which is a perfect square.\\
			Therefore, if 1 is added to a product of twin primes, a perfect square is always obtained
		\end{proof}
	\item[3.3.02b] Show that the sum of twin primes $ p $ and $ p+2 $ is divisible by 12, provided that $ p>3 $.
	\begin{proof}
		Let $ p $ and $ p+2, p>3 $ be twin primes. Then we can rewrite $ p $ as $ 6k+1 $ and $ 6k-1 $ for some $ k\in\Z $. Then, 
		\[p+p+2=6k+1+6k-1=12k|12\]
		Therefore, the sum of twin primes $ p $ and $ p+2 $ is divisible by 12, provided that $ p>3 $. 
	\end{proof}
		
	\item[3.3.06] Prove that the Goldbach conjecture that every integer greater than 2 is the sum of two primes is equivalent to the statement that every integer greater than 5 is the sum of three primes.
	\begin{proof}
		Let $ p_1 $ and $ p_2 $ be primes. Let $ n\in \Z >2 $ then $ 2n-2>2 $. Then,
		\[2n-2=p_1+p_2\Rightarrow 2n=p_1+p+2+2\]
		But $ n>2\Rightarrow 2n>4 \Rightarrow 2n+1>5 $. So, $ 2n+1=p_1+p_2+3 $.\\
		Therefore, every integer greater than 2 is the sum of two primes is equivalent to the statement that every integer greater than 5 is the sum of three primes
	\end{proof}
	
	\item[3.3.24] Determine all twin primes $ p $ and $ q = p+2 $ for which $ pq-2 $ is also prime.
		\begin{proof}
			Let $ p $ and $ q $ be twin primes. If $ p=3 $ and $ q=5 $ then $ pq-2 =13$	which is prime. Suppose that $ p>3 $. Then $ p=6k+1 $ or $ p=6k+5 $. If $ p=6k+1 \Rightarrow p+2 = 6k+3 $ which is composite. Thus $ p=6k+5 \Rightarrow p+2 = 6k+7 $. Notice.
			\[pq-2=(6k+5)(6k+7)-2=36k^2+72k+33=3(12k^2+24k+11)\] which is divisible by 3.\\
			Thus, 3 and 5 are the only twin primes for which $ pq-2 $ is prime.
		\end{proof}
	\item[3.3.28a] If $ n>1 $, show that $ n! $ is never a perfect square.
		\begin{proof}
			Let $ n>1 $.
			\begin{enumerate}
				\item [Case 1:] $ n $ is prime. Then for $ n! $ to be a perfect square one of $ n-1,n-2,\cdots,2 $ must contain n as a factor. But this means one of $n-1,n-2,\cdots,2 \geq n $ which is impossible.
				\item[Case 2:] $ n $ is not prime. Then the first prime less than $ n $ is for all $p,k\in\Z, p = n-k,0<k<n-1,2\leq p < n $ No number less than $ p $ will contain $p$ as a factor. Thus, for $ n! $ to be a perfect square there exists a multiple of $ p $, called $ bp, 1<b<n$, such that $p<bp≤n$. There must exists a prime number between $ p $ and $ 2p $. Then if $ r<n<2r $ and also $ p<n  $, so such an$  n! $ would never be a perfect square.
			\end{enumerate}
		\end{proof}
	\item[3.3.28b] Find the values of $ n\geq1 $ for which
		\[n!+(n+1)!+(n+2)!\]
		is a perfect square.
		\begin{proof}
			Let $ n \geq 1 $. Notice.
			\[n!+(n+1)!+(n+2)!=n!(1+(n+1)+(n+1)(n+2))=n!(n+2)^2\]
			As $ n! $ is never a perfect square, the only $ n $ for which \[n!+(n+1)!+(n+2)!\] is a perfect square is when $ n=1 $.
		\end{proof}
\end{enumerate}

\end{document}
