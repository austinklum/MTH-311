%This is a Latex file.
\documentclass[12pt]{article}
\usepackage{latexsym,fancyhdr,amsmath,amsfonts,amsthm,dsfont}
\usepackage{amssymb}

% margins are relative to the default of 1 in
%\topmargin       -0.2 in

\topmargin        -0.2 in
\textheight       8.4 in
\oddsidemargin    0 in     % this is for pages 1, 3, 5, ...
\evensidemargin   0 in     % and this for 2, 4, 6, ...
\textwidth        6.5 in
\headheight       0 in     % we won't have a running head, nor
\headsep          .35 in     % any extra space between head and text

%\parindent 0pt

\pagestyle{fancy} \lhead{\sf MTH 311} \chead{\sf Homework \#03}
\rhead{\sf Austin Klum} \lfoot{} \cfoot{} \rfoot{}

\newcommand{\C}{\mathds{C}}
\newcommand{\I}{\mathds{I}}
\newcommand{\N}{\mathds{N}}
\newcommand{\Q}{\mathds{Q}}
\newcommand{\R}{\mathds{R}}
\newcommand{\Z}{\mathds{Z}}

\begin{document}
\begin{enumerate}
	\item[2.3.3] Prove or Disprove: if $a | (b+c)$, then either $a | b$ or $a | c$.\\
		Let $a=2,b=1,$ and $c=1$. Then,
			\[a|(b+c) \text{ as } 2 | (1+1)=2\]
		But $2\not|$ 1. Therefore by counterexample the original statement must be false.
	\item[2.3.5] Prove that for any integer $a$, one of the integers $a, a+2,a+4$ is divisible by 3.
		\begin{proof}
			From the division algorithm we have $a=3n+0,a=3n+1,a=3n+2$ for some $n \in \Z$.
			\begin{enumerate}
				\item [Case 1:] Let $a=3n+0$. Then, a is divisible by 3 by definition.
				\item [Case 2:] Let $a=3n+1$. Then, 
					\[a+2 = 3n+3 = 3(n+1)\]
					Thus, $a+2$ is divisible by 3.
				\item[Case 3:] Let $a=3n+2$. Then,
					\[a+4=3n+6=3(n+2)\]
					Thus, $a+4$ is divisible by 3. 
			\end{enumerate}
			Therefore, for any integer $a$, one of the integers of the form $a, a+2,a+4$ is divisible by 3.
		\end{proof}
	\item[2.3.9] Establish that the difference of two consecutive cubes is never divisible by 2.
		\begin{proof}
			Let the difference of two consecutive cubes be represented by $(n+1)^3-(n)^3$ for some $n\in \Z$.
			As we know the two cubes are consecutive, one of the cubes must be even.
			\begin{enumerate}
				\item [Case 1:] The smaller cube is even. Then, $n$ is even. That is to say, there exists a $k\in\Z$ such that $n=2k$ and $n+1=2k+1$. Notice.
				\begin{align*}
					(n+1)^3-(n)^3 &= (2k+1)^3-(2k)^3 \\
								  &= 12k^2+6k+1 \\
								  &= 2(6k^2+3k)+1\\
								  \text{ Let } s:=6k^2+3k \\
								  &= 2s+1 \quad \text{ which is odd}			 	
				\end{align*}
				\item[Case 2:] The smaller cube is odd. Then, $n$ is odd. That is to say, there exists a $k\in\Z$ such that $n=2k+1$ and $n+1=2k+2$. Notice.
				\begin{align*}
					(n+1)^3-(n)^3 &= (2k+2)^3-(2k+1)^3 \\
								  &= 12k^2+18k+7 \\
								  &= 2(6k^2+9k+3)+1\\
								  \text{ Let } s:=6k^2+9k+3 \\
								  &= 2s+1 \quad \text{ which is odd}			 	
				\end{align*}
			\end{enumerate}
			As we have shown that in both cases the result from the difference of two consecutive cubes is odd. Therefore the difference of two consecutive cubes is never divisible by 2.
		\end{proof}
	\item[2.3.15] If $a$ and $b$ are integers, not both of which are zero, prove that $\gcd(2a-3b,4a-5b)$ divides $b$; hence, $\gcd(2a+3,4a+5) = 1$.
		\begin{proof}
			$\gcd(2a-3b,4a-5b) = x(2a-3b) + y(4a-5b)$ for some $x,y \in\Z$. Let $x=-2$ and $y=1$. Then,
				\[-2(2a-3b) + 1(4a-5b) = -4a+6b+4a-5b = b\]
			Thus, $\gcd(2a-3b,4a-5b)|b$.\\
			\\
			Continuing, \\
			$\gcd(2a+3,4a+5) = x(2a+3) + y(4a+5) = 1$ for some $x,y\in\Z$. Let $x=2$ and $y=-1$. Then,
				\[2(2a+3)-1(4a+5) = 4a+6-4a+5 = 1\]
			As $\gcd(2a+3,4a+5)|1 \Rightarrow \gcd(2a+3,4a+5) = 1$\\
			\\
			Therefore, both $\gcd(2a-3b,4a-5b)$ divides $b$; hence, $\gcd(2a+3,4a+5) = 1$ are true.
		\end{proof}
	\item[2.3.17] Prove that the expression $(3n)!/(3!)^n$ is an integer for all $n\geq 0$.
		\begin{proof}
			In the case of $n=1$, We have $3!/3!^1 = 6/6 = 1$. So, the expression hold true for $n=1$. Assume that the expression holds true for some $k\geq0$, that is $n=k$, $(3k)!/(3!)^k$. We will now show that the expression holds true for $k+1$. We see that
			\begin{align*}
				\frac{[3(k+1)]!}{(3!)^{k+1}} &= \frac{(3k+3)!}{(3!)^k3!}\\
											 &= \frac{(3k)!}{3!^k}\cdot\frac{(3k+3)(3k+2)(3k+1)}{6}\\
											 \text{Let } q:=\frac{(3k)!}{3!^k}\\
											 &= q\cdot\frac{(k+1)(3k+2)(3k+1)}{2} \\
			\end{align*}
			As $q$ must be an integer, we need to prove that $\frac{(k+1)(3k+2)(3k+1)}{2}$ is also an integer.
			\begin{enumerate}
				\item[Case 1:] $k$ is even. Then, for some $r\in\Z, k=2r \Rightarrow 3k+2=6r+2=2(3r+1)$. Thus, $ 2 | 3k+2$.
				\item[Case 2:] $k$ is odd. Then, for some $r\in\Z, k=2r+1 \Rightarrow k+1=2r+2=2(r+1)$. Thus, $ 2 | k+1$.
			\end{enumerate}
			Thus, $\frac{(k+1)(3k+2)(3k+1)}{2}$ is an integer. Which means $\frac{[3(k+1)]!}{(3!)^{k+1}}$ must also be an integer.
			Therefore, by mathematical induction, $(3n)!/(3!)^n$ is an integer for all $n\geq 0$.
		\end{proof}
	\item[2.3.23] If $a | bc$ show that $a | \gcd(a,b) \gcd(a,c)$
    	\begin{proof}
    		Assume $a | bc$. Then, $bc = aq$ for some $q\in\Z$. By definition, $\gcd(a,b)=am+bn$ for some $m,n\in\Z$ and $\gcd(a,c)=ar+bs$ for some $r,s\in\Z$. Then,
    			\begin{align*}
    				\gcd(a,b)\gcd(a,c)&=(am+bn)(ar+bs)\\
    								  &=a^2mr+acms+abrn+bcns\\
    								  &=a^2mr+acms+abrn+aqns, \text{ as } bc = aq\\
    								  &= a(amr+cms+brn+qns)
    			\end{align*} 
    			Therefore, $a | \gcd(a,b)\gcd(a,c)$
    	\end{proof}
        
\end{enumerate}


\end{document}
